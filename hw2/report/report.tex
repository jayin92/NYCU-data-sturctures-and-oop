\documentclass{article}[12pt]
\usepackage{fontspec}   %加這個就可以設定字體
\usepackage{xeCJK}       %讓中英文字體分開設置
\usepackage{indentfirst}
\usepackage{listings}
\usepackage[newfloat]{minted}
\usepackage{float}
\usepackage{graphicx}
\usepackage{caption}
\usepackage{fancyhdr}
\usepackage{hyperref}
\usepackage{amsmath}
\usepackage{multirow}
\usepackage[dvipsnames]{xcolor}
\usepackage{graphicx}
\usepackage{tabularx}
\usepackage{booktabs}
\usepackage{caption}
\usepackage{subcaption}
\usepackage{pifont}
\usepackage{amssymb}
\usepackage{upquote}

\usepackage[breakable, listings, skins, minted]{tcolorbox}
\usepackage{etoolbox}
\setminted{fontsize=\scriptsize}
\renewtcblisting{minted}{%
	listing engine=minted,
	minted language=cpp,
	listing only,
	breakable,
	enhanced,
	minted options = {
		linenos, 
		breaklines=true, 
		breakbefore=., 
		fontsize=\scriptsize, 
		numbersep=2mm
	},
	overlay={%
		\begin{tcbclipinterior}
			\fill[gray!25] (frame.south west) rectangle ([xshift=4mm]frame.north west);
		\end{tcbclipinterior}
	}   
}

\usepackage[
top=2cm,
bottom=2cm,
left=3.5cm,
right=3.5cm,
headheight=17pt, % as per the warning by fancyhdr
includehead,includefoot,
heightrounded, % to avoid spurious underfull messages
]{geometry} 

\newenvironment{code}{\captionsetup{type=listing}}{}
\SetupFloatingEnvironment{listing}{name=Code}


\title{%
	資料結構與物件導向程式設計 \\
	\Large HW2 報告}
\author{110550088 李杰穎}
\date{\today}

\setCJKmainfont{Noto Serif TC}
\setmonofont[Mapping=tex-text]{Consolas}

\XeTeXlinebreaklocale "zh"             %這兩行一定要加,中文才能自動換行
\XeTeXlinebreakskip = 0pt plus 1pt     %這兩行一定要加,中文才能自動換行

\setlength{\parindent}{2em}
\setlength{\parskip}{1em}
\renewcommand{\baselinestretch}{1.25}




\begin{document}
\maketitle
\section{Part 1: Directed Graph}

此部分是給定一個有向圖,這個有向圖可能是無環也有可能是有環的,所以我們必須先判斷給定的圖是否有環。若有環,則我們必須把圖上的強連通分量找出來。若無環,則我們要印出這張有向圖的拓樸排序。

\subsection{Implementation Detail}

以下是 \texttt{PartI.h},內有 Part 1 會用到的所有函數及變數。

\begin{code}
\captionof{listing}{\texttt{PartI.h}}
\begin{minted}
#ifndef PARTI_H
#include "SolverBase.h"
#include <algorithm>
#include <vector>
#include <map>
#include <iostream>
#include <fstream>

using namespace std;

class PartI : public SolverBase
{
int n, m, cnt;
vector<vector<pair<int, int>>> graph; // Declare an adjacency list to save the graph (pair fisrt: vertex second: weight)
vector<vector<pair<int, int>>> rev_graph; // Declare an adjacency list to save the "reverse" graph (pair fisrt: vertex second: weight)
map<pair<int, int>, int> scc_graph;  // Use map to save edges and theirs weights
vector<pair<int, vector<int>>> scc_vertex; // Save the id of each vertices in every SCC
vector<int> order; // Save topological order
vector<int> scc; // Save the id of corresponding SCC of each vertex

map<int, int> finish; // Store the status of vertex when traversing
bool isAyclic = true; // To store that this graph is acyclic or not

public:
    void read(std::string); // Read input from file
    void solve(); // Main solve function
    void write(std::string); // Write output to file
    void dfs(int); // DFS
    void scc_dfs(int, int); // Run dfs to get SCC
    void scc_revdfs(int); // Run reverse dfs to get SCC
    void kosaraju(); // The main part of Kosaraju's Algorithm
    void buildGraph(); // Build coarse graph
    void buildGraphDFS(int); // Use dfs to build the coarse graph
};

#define PARTI_H
#endif
\end{minted}
\end{code}

以下為 Part 1 的具體實作,其細節我都已經寫在註解中。主要就是先利用 DFS 判斷圖是否有環,若判斷無環,則根據先前 DFS 的結果輸出 topological ordering。若判斷有環,則會利用 Kosaraju 演算法來找出強連通分量。

Kosaraju's algorithm 主要演算步驟如下:
\begin{enumerate}
\item 對有向圖 $G$ 取反圖,得到 $G$ 的反向圖 $G^{R}$
\item 利用 DFS 找出 $G^R$ 的拓樸排序
\item 對 $G$ 按照拓樸排序進行 DFS
\item 在同一次 DFS 跑到的點都在同一個強連通分量中
\end{enumerate}

以下就是 Part 1 實作的所有程式碼,程式碼的解釋部分我已經寫在註解。

\begin{code}
\captionof{listing}{\texttt{PartI.cpp}}
\begin{minted}
#include "PartI.h"

using namespace std;

bool operator<(const pair<int, int>& a, const pair<int, int>& b){
    return (a.first < b.first) || (a.first == b.first && a.second < b.second);
}

void PartI::read(string file) {
    cout << "Part I reading..." << endl;
    ifstream ifs(file); // Use input file stream to read input from file
    ifs >> n >> m; // Get # of vertices n and # of edges m
    // Resize the vector
    graph.resize(n);
    rev_graph.resize(n);
    scc.resize(n);
    int u, v, w;

    // Build graph 
    for(int i=0;i<m;i++){
        ifs >> u >> v >> w;
        graph[u].push_back({v, w});
        rev_graph[v].push_back({u, w});
    }
    // Close the ifstream
    ifs.close();

    // Because we need to traverse from smallest index to the largest, thus we need to sort the adj. list
    for(auto i: graph){
        sort(i.begin(), i.end());
    }
}

void PartI::solve() {
    cout << "Part I solving..." << endl;
    for(int i=0;i<n;i++){
        // Run DFS for not yet traversed vertex
        if(finish[i] == 0){
            dfs(i);
        }
    }

    if(isAyclic){
        // If is acylic, then reverse the order to get real topological ordering
        reverse(order.begin(), order.end());
        return;
    }
    
}

void PartI::write(string file) {
    cout << "Part I writing..." << endl;
    ofstream ofs(file); // use output filestream to write output to file
    if(isAyclic){
        // Write the topological ordering to file
        for(auto i: order){
            ofs << i << " ";
        }
        ofs << endl;
        // After writing the to the file, directly exit the function
        return;
    }
    // If not ayclic, then run Kosaraju's algorithtm to find SCC
    kosaraju();
    // After running Kosaraju's algorithtm, vertex will have a label which indicates that which SCC it belongs to.
    // This information will save in vector<int> scc.
    // cnt is # of SCC in the given graph.
    scc_vertex.resize(cnt, {1e9, {}}); // scc_vertex will save the minimum index and every vertex in each SCC.
    // Iterate all vertices in the graph.
    for(int i=0;i<n;i++){
        int u = scc[i];
        scc_vertex[u].first = min(scc_vertex[u].first, i);
        scc_vertex[u].second.push_back(i); 
    }
    sort(scc_vertex.begin(), scc_vertex.end()); // Sort the SCCs based on the minimum index.

    // Relabel the vertices to match the new index value of every SCC
    for(int i=0;i<cnt;i++){
        for(auto j: scc_vertex[i].second){
            scc[j] = i;
        }
    }

    // Build coarse graph
    buildGraph();
    // After running the above function, we will get the coarse graph and save as map.

    ofs << cnt << " " << scc_graph.size() << endl; // Output the # of vertcies and edges

    // Because map will sort by key, we don't need to sort the map.
    for(auto i: scc_graph){
        ofs << i.first.first << " " << i.first.second << " " << i.second << endl;
    }

    // Close output filestream
    ofs.close();
    return;
}

void PartI::dfs(int v){
    // Back edge, cyclic
    if(finish[v] == 1){
        isAyclic = false;
        return;
    }

    // forward edge or cross edge
    if(finish[v] == 2){
        return;
    }

    // mark 1
    finish[v] = 1;
    for(auto i: graph[v]){
        dfs(i.first);
    }

    // mark 2
    finish[v] = 2;
    order.push_back(v);
}

void PartI::scc_revdfs(int v){
    // DFS runs on reverse graph
    finish[v] = 1;
    for(auto i: rev_graph[v]){
        if(finish[v] == 0){
            scc_revdfs(i.first);
        }
    }
    order.push_back(v);
}

void PartI::scc_dfs(int cur, int s){
    // DFS runs on graph and lable the vertices
    scc[cur] = s;
    for(auto i: graph[cur]){
        int u = i.first;
        if(scc[u] == -1){
            scc_dfs(u, s);
        }
    }
}

void PartI::kosaraju(){
    order.clear();  // Clear the original order
    finish.clear(); // Clear the vector
    scc.resize(n);  // Resize the vector that store vertex belongs to SCC
    fill(scc.begin(), scc.end(), -1); // Fill the vector with -1
    for(int i=0;i<n;i++){
        if(finish[i] == 0){ // If not yet traversed, then run DFS on reverse graph.
            scc_revdfs(i);
        }
    }
    cnt = 0; // Store the # of SCC
    reverse(order.begin(), order.end());
    // Use topological ordering of reverse graph to get SCC
    for(auto i: order){
        if(scc[i] == -1){    // Not in any SCC
            scc_dfs(i, cnt); // Run DFS
            cnt ++;
        }
    }
}

void PartI::buildGraphDFS(int v){
    finish[v] = 1;
    for(auto i: graph[v]){ // Iterate all edges that connect to vertex v
        int u = i.first;
        if(scc[v] != scc[u]){
            scc_graph[{scc[v], scc[u]}] ++; // If two vertices belongs to different SCC, then this edge will appear in coarse graph and weight will plus 1
        }
        if(finish[u] == 0){ // If not yet visited, then run DFS
            buildGraphDFS(u);
        }
    }
}

void PartI::buildGraph(){
    finish.clear(); // Clear the vector that records visited vertices
    for(int i=0;i<n;i++){
        if(finish[i] == 0){
            buildGraphDFS(i);
        }
    }
}
\end{minted}
\end{code}

\subsection{Discussion}
\subsubsection{Time Complexity}

此演算法時間複雜度的來源主要是多次 DFS,而一次 DFS 的時間複雜度為 $O(V+E)$,其中 $V$ 為點的數量、$E$ 為邊的數量。而 Kosaraju 演算法的複雜度也只是 2 次 DFS。其中還有用 map 維護 \texttt{finish},其插入和查詢的總時間複雜度為 $O(V \log V)$。故總時間複雜度應為 $O(V + V \log V + E)$。

\subsubsection{Your Discovery}

在本次作業中,我發現可以僅需要一次 DFS 就可以得到 topological ordering,且 DFS 的起始點不一定要是入度為 0 的點,從任一點開始,只要 traverse 所有點,就可以得到正確的 topological ordering。

\subsubsection{Which is the better algorithm in which condition}

事實上,其實有另一個演算法可以計算強連通分量,此演算法即為 Tarjan 演算法。Tarjan 演算法雖然在時間複雜度上與 Kosaraju 演算法相同,但其常數部分差了一倍。Tarjan 演算法只需要一次 DFS 即可以找到強連通分量。但 Tarjan 演算法實作上比較麻煩,故在本次作業中,我仍採用 Kosaraju 演算法。

\subsubsection{Encountered Problems}

一開始在寫作業時沒有很理解 \texttt{read} 函數中 string 引數所代表的意義,後來查看 \texttt{main.cpp} 後才發現原來是檔名,於是想到可以用 C++ 內建的 \texttt{fstream} 來開啟檔案,並利用與 \texttt{cin, cout} 相同的方式來將變數輸入進來,也可以將變數輸出到檔案中。

\section{Part 2: Shortest Path}

在此部分中,我們需要實作兩種圖上最短路演算法,分別為 Dijkstra 演算法及 Bellman-Ford 演算法。

Dijsktra 演算法僅能處理無負邊權的圖,而 Bellman-Ford 演算法可以處理負邊權的圖,也可以偵測出圖上的負環。

在本次作業中,Dijkstra 演算法的部分我們會將輸入的邊權取絕對值,使圖上不會出現負邊權的情況。


\subsection{Implementation Detail}

以下為 \texttt{PartII.h},有本部分會用到的函數及變數。

\begin{code}
\captionof{listing}{\texttt{PartII.h}}
\begin{minted}
#ifndef PARTII_H
#include "SolverBase.h"
#include <iostream>
#include <fstream>
#include <algorithm>
#include <vector>
#include <queue>

using namespace std;

class PartII : public SolverBase
{
    int n, m; // # of vertices and edges
    bool hasNegCyc = false; // has negative cycle in the graph or not
    vector<vector<pair<int, int>>> adj; // Declare an adjacency list to save the graph (pair fisrt: vertex second: weight)
    vector<vector<pair<int, int>>> abs_adj; // Declare an adjacency list to save the graph (pair fisrt: vertex second: abs(weight))
    vector<int> d, abs_d; // The shortest distance from node 0 to every other vertices
public:
    void read(std::string); // Read input from file
    void solve(); // Main solve fucntion
    void write(std::string); // Write the output to file
    void dijkstra(int); // Dijkstra's Algorithm
    bool bellmanFord(int); // Bellman-Ford's Algorithm
};

#define PARTII_H
#endif
\end{minted}
\end{code}

以下就是 Part 2 的所有程式碼,解釋的部分已放在註解中。

\begin{code}
\captionof{listing}{\texttt{PartII.cpp}}
\begin{minted}
#include "PartII.h"

#define INF 1e9

using namespace std;

void PartII::read(string file) {
    cout << "Part II reading..." << endl;
    ifstream ifs(file); // Decalre a input filestream to get input from file
    ifs >> n >> m; // input # of vertices and edges
    // Resize the vectors
    d.resize(n);
    abs_d.resize(n);
    adj.resize(n); 
    abs_adj.resize(n);
    
    // Build graph from input
    int u, v, w;
    for(int i=0;i<m;i++){
        ifs >> u >> v >> w;
        adj[u].push_back({v, w});
        abs_adj[u].push_back({v, abs(w)});
    }
     
    ifs.close();
}
void PartII::solve() {
    cout << "Part II solving..." << endl;
    dijkstra(0); // Run from Dijkstra's algorithm from node 0

    // Run Bellman-Ford algorithm from node 0.
    // And if function return false, it means there is negative cycle in the graph
    hasNegCyc = !bellmanFord(0); 
}

void PartII::write(string file) {
    cout << "Part II writing..." << endl;
    ofstream ofs(file);
    ofs << abs_d[n-1] << endl; // Output the Dijkstra algorithm's output

    if(hasNegCyc){
        ofs << "Negative loop detected!" << endl; // If has negative cycle, then output "Negative loop detected!"
    } else {
        ofs << d[n-1] << endl; // Otherwise, output the Bellman-Ford algorithm's output
    }

    ofs.close(); // Close output file stream
}


void PartII::dijkstra(int src){
    fill(abs_d.begin(), abs_d.end(), INF); // Fill the distance vector w/ INF value
    abs_d[src] = 0; // The distance from node 0 to node 0 is 0

    // Use priority_queue to get the vertex with the smallest distance from node 0 in every step
    priority_queue<pair<int, int>, vector<pair<int, int>>, greater<pair<int, int>>> pq; // first: weight, second: vertex

    // Push the initial node 0 to pq
    pq.push({0, 0});

    // If pq is not empty, keep running the relax process
    while(!pq.empty()){
        auto edge = pq.top();
        pq.pop();
        int u = edge.second;
        for(auto i: abs_adj[u]){
            int w = i.first;
            int v = i.second;
            int alt = abs_d[u] + i.first; // Alternate path's distance

            // If alternate distance less than current distance save in distance vector
            if(alt < abs_d[v]){
                abs_d[v] = alt;    // Update the distance in distance vector
                pq.push({alt, v}); // Update the new distance to pq
            }
        }
    }
}

bool PartII::bellmanFord(int src){
    fill(d.begin(), d.end(), INF); // Fill the distance vector w/ INF value
    d[src] = 0; // The distance from node 0 to node 0 is 0

    // Run n-1 times of relaxation process
    for(int i=0;i<n-1;i++){
        for(int u=0;u<n;u++){
            for(auto k: adj[u]){
                int v = k.first;
                int w = k.second;
                int alt = d[u] + w;

                // If alternate distance less than current distance save in distance vector
                if(alt < d[v]){
                    d[v] = alt; // Update the distance in distance vector
                }
            }
        }
    }


    for(int u=0;u<n;u++){
        for(auto k: adj[u]){
            int v = k.first;
            int w = k.second;
            int alt = d[u] + w;
            if(alt < d[v]){
                // If there still have edge that can be relaxed in nth times of relaxation process.
                // Then there exists a negative cycle
                return false;
            }
        }
    }

    return true;
}
\end{minted}
\end{code}

\subsection{Discussion}
\subsubsection{Time Complexity}

以 Dijkstra 演算法來說,一般來說其時間複雜度為 $O(V^2)$,其中 $V$ 為點的數量。但因為我在本次作業中使用 \texttt{priority\_queue} 來加速取出最小值的過程,而 \texttt{priority\_queue} 本質上是一個 Binary Heap,所以 Dijsktra 演算法的時間複雜度降至 $O((E+V)\log V)$。

至於 Bellman-Ford 演算法,其時間複雜度即為 $O(E \cdot V)$。

\subsubsection{Your Discovery}

在實作 Dijkstra 演算法中的 relaxation process 時,雖然實際上我們要做的應該是去更新 \texttt{priority\_queue} 中的已存在的節點,但實際上,我們只要把更小的值推進去就好,因為 \texttt{priority\_queue} 永遠會把較小的值拿出來,等到較大值拿出來時,這個較大的值也不會對 distance vector 產生更新。

\subsubsection{How to find path of the shortest path?}

我們可以在程式中額外宣告一個 \texttt{parent} 陣列,此陣列是要記錄最短路徑中,每一個節點的前一個節點為何,我們只需要在 relaxing 的時候更新這個陣列即可。跑完兩種演算法後,我們從終點 backtracking 回到起點,最後就可以找到最短路徑了。

\subsubsection{How Bellman-Ford Algorithm detect negative cycle?}

因為在圖中,我們可以發現最短路徑最多就經過 $V-1$ 條邊,所以我們對所有邊 relaxing $V-1$ 次必定能找到最短路徑。但如果圖內有 negative loop 則此張圖就沒有最短路徑,因為我們可以透過重複經過此 negative loop 來使最短距離越來越小。而 Bellman-Ford 演算法即是在跑完 $V-1$ 迭代過程後再額外跑一次迭代,如果有任何一條邊可以被 relaxing 到,則此圖中就有 negative loop。

\subsubsection{Which is the better algorithm in which condition?}

如上所述 Dijkstra 演算法無法處理有負邊的情況,也無法判斷是否有負環。而 Bellman-Ford 演算法可以處理負邊也可以處理負環。但 Dijkstra 演算法的整體複雜度較佳,故如果確定圖上沒有負邊,則我們應該優先考慮使用 Dijkstra 演算法,但如果不確定是否有負邊,則我們應該使用 Bellman-Ford 演算法。

\end{document}
